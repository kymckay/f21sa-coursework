\documentclass[11pt]{article}
\usepackage[margin=2cm]{geometry}

% for links
\usepackage{hyperref}

% to include another PDF
\usepackage{pdfpages}
% \includepdf[fitpaper]{map.pdf}

% to include images
\usepackage{graphicx}

% \begin{center}
% 	\includegraphics[clip,width=.8\textwidth]{figures/uml-use-case}
% \end{center}

% for equation environments
\usepackage{amsmath}

% for code snippets
\usepackage{listings}

% \begin{lstlisting}[language=Python]
% \end{lstlisting}

% \lstinputlisting[language=Octave]{BitXorMatrix.m}

% For header and footer
\usepackage{fancyhdr}
\pagestyle{fancy}
\fancyhf{} % clears default style
\lhead{F21SA Statistical Modelling and Analysis}
\lfoot{Kyle Mckay (km2008)}
\cfoot{\thepage}
\rfoot{HWU Person ID: H00358352}
\renewcommand{\headrulewidth}{0pt}

% No paragraph indentation
\setlength\parindent{0pt}
\setlength\parskip{1em}
\raggedright


\begin{document}

\begin{center}
    \Large{F21SA Assessed Project}
\end{center}

\section{Introduction}

Average wind speeds, $\underline{x}$, at the top of Arthur's Seat on 1000
days between 2018-2020 are analysed to produce a Rayleigh statistical
model in order to make predictions of future wind speeds.

Maximum likelihood estimation of the distribution's scale parameter,
$\sigma$, is used to fit the model to the data so that simulation of
future wind speeds can be performed. The probability of predicted future
wind speeds having a greater scale parameter than the measured data is
then used to determine if the predictive model is robust to errors in
the estimation of $\sigma$.

% TODO conclusions

\section{Data Inspection}

\begin{figure}[ht]
    \centering
	\includegraphics[clip]{DailyMean.pdf}
    \caption{Histogram of 1000 daily average wind speeds on Arthur's Seat}
    \label{fig:histogram}
\end{figure}


\begin{table}[ht]
    \centering
    \begin{tabular}{ll}
    Statistic & Value \\ \hline
    Mean, $\underline{\bar x}$ & 12.945 mph \\
    Median & 12.145 mph \\
    Standard deviation, $sd(\underline{x})$ & 6.997 mph \\
    5-quantiles & [0.13, 7.615, 12.145, 16.96, 41.87] mph
    \end{tabular}
    \caption{Summary statistics of daily average wind speeds on Arthur's Seat}
    \label{t:data}
\end{table}

The mean is found to be greater than the median (see Table \ref{t:data})
which means the wind data has a positive skew, but overall is not very dispersed
since the standard deviation is only $\sim17\%$ of the range.

\section{Scale Parameter Estimation}

Considering the likelihood function of the Rayleigh model,

$$
    L(\sigma) = \prod_{i=1}^n{ \frac{x_i}{\sigma^2} \exp\left( \frac{-x_i^2}{2 \sigma^2} \right)}
    = \frac{1}{\sigma^{2n}} \times \prod_{i=1}^n{ x_i \exp\left( \frac{-x_i^2}{2 \sigma^2} \right)} \text{,}
$$

the log likelihood function is used to simplify calculations (since the function minimums occur at the same $\sigma$),

$$
    l(\sigma) = \log\left(\frac{1}{\sigma^{2n}}\right) + \sum_{i=1}^n{ \log(x_i) - \frac{x_i}{2 \sigma^2} }
    = -2n \log(\sigma) + \sum_{i=1}^n{\log(x_i)} - \frac{1}{2 \sigma^2} \sum_{i=1}^n{x_i^2} \text{.}
$$

The score function is needed, which is the first derivative of log likelihood,

$$
    U(\sigma) = \frac{\partial l}{\partial \sigma} = \frac{-2n}{\sigma} + \frac{1}{\sigma^3} \sum_{i=1}^n{x_i^2} \text{.}
$$

The maximum likelihood estimator (denoted by $ \hat\sigma $) is found by equating the score function to 0 to find where likelihood is minimised
by solving for $ \hat\sigma $.

\begin{align*}
    \frac{-2n}{\hat{\sigma}} + \frac{1}{\hat{\sigma}^3} \sum_{i=1}^n{x_i^2} &= 0 \\
    -2n \hat\sigma^2 + \sum_{i=1}^n{x_i^2} &= 0 \\
    \hat\sigma = \sqrt{\frac{\sum_{i=1}^n{x_i^2}}{2n}} \text{.}
\end{align*}

\subsection{Estimator Distribution}

The Fisher information is found as

\begin{align*}
I(\sigma) &= - E\left[ \frac{\partial^2 l}{\partial \sigma^2}(\sigma)\right]
= - E\left[ \frac{2n}{\sigma^2} - \frac{3}{\sigma^4} \sum_{i=1}^n {x_i^2} \right] \\
&= - \frac{2n}{\sigma^2} + \frac{3}{\sigma^4} \sum_{i=1}^n {E(x_i^2)}
= - \frac{2n}{\sigma^2} + \frac{6n}{\sigma^2} = \frac{4n}{\sigma^2}
\end{align*}

For large $ n $, $ \hat\sigma $ is approximately distributed as $ N(\sigma, \frac{1}{I(\sigma)}) $. In this case that is $ N(\sigma, \frac{\sigma^2}{4n}) $.

\subsection{Estimator Confidence}

For the wind data $ \hat\sigma $ is calculated in R (see Appendix) to be 10.40396 mph.

To obtain a 95\% equal-tailed confidence interval,

\begin{align*}
    I_{0.95} = \hat\sigma &\pm \left(z_{0.025} \times ese(\hat\sigma)\right)
    \intertext{where}
    ese(\hat\sigma) &= \sqrt{1/I(\hat\sigma)} = \sqrt{\hat\sigma^2/4n} \text{,}
\end{align*}

calculation is performed in R (with $z_{0.025} = 1.96$ taken from NCST Table 5) to obtain the result

$$ I_{0.95} = [ 10.08153 , 10.72638 ] \text{.}$$

\section{Future Wind Speed Simulation}

\begin{figure}[ht]
    \centering
	\includegraphics[clip]{PredictionMean.pdf}
    \caption{Histogram of 10,000 predicated mean wind speeds in next 1000 days}
    \label{fig:mean_hist}
\end{figure}

\begin{table}[ht]
    \centering
    \begin{tabular}{ll}
    Statistic & Value \\ \hline
    Mean, $\bar Y^\prime$ \rule{0pt}{3ex} & 13.041 mph \\
    Median & 13.041 mph \\
    Standard deviation, $sd(Y^\prime)$ & 0.216 mph \\
    5-quantiles & [12.21, 12.89, 13.04, 13.18, 13.91] mph
    \end{tabular}
    \caption{Summary statistics of 10,000 predicted mean wind speeds in next 1000 days}
    \label{t:mean}
\end{table}

It seems from simulation in R (see Figure \ref{fig:mean_hist}) that the predicted mean
wind speed in the next 1000 days has an approximate Normal distribution.
This is corroborated by the equivalent mean and medians in Table
\ref{t:mean} -- the entries of which suggest that
$Y^\prime \sim N(13.041, 0.216^2)$. A normal distribution is expected as
per the central limit theorem.

\subsection{Increasing Wind Speed Probability}

From the central limit theorem, the population standard deviation is

\begin{align*}
    \sigma = \sqrt{Var(Y^\prime) \times n} = \sqrt{0.216^2 \times 1000} \approx 6.831 \text{,}
\intertext{which is used to find a pivotal value for the point of interest (when $S = sd(\underline{x})$)}
    \frac{(n-1)sd(\underline{x})^2}{\sigma^2} \sim \chi_{n-1}^2
\end{align*}

In R (see Appendix) this is found to be approximately 1043.5 and used with the chi-squared cumulative distribution function to obtain

$$ P[S > sd(\underline{x})] = 0.1597 $$

\subsection{Conclusion Robustness}

Repeating the same simulation and calculations as above for the lower and upper estimates, $\sigma_L(\underline{x})$ and $\sigma_U(\underline{x})$ respectively, gives

\begin{align*}
    P[S > sd(\underline{x})]_L &= 0.0038 \\
    P[S > sd(\underline{x})]_U &= 0.6974
\end{align*}

\newpage
\section*{Appendix}
\subsection*{investigation.r}
\lstinputlisting[language=R]{investigation.r}

\subsection*{simulate.r}
\lstinputlisting[language=R]{simulate.r}

\subsection*{probability.r}
\lstinputlisting[language=R]{probability.r}

\end{document}
